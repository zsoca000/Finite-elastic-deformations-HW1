\documentclass[12pt,a4paper]{article}

\setcounter{secnumdepth}{3}
% USEPACKAGE LISTA
\usepackage[utf8]{inputenc}
\usepackage{amsmath}
\usepackage{mathtools}
\usepackage{marvosym} 
\usepackage{wrapfig}
\usepackage{hyperref}
\usepackage{float}
\usepackage{multicol}
\hypersetup{colorlinks,citecolor=black,filecolor=black,linkcolor=black,urlcolor=black}
\usepackage{pdfpages}
\usepackage{amsfonts}
\usepackage{amssymb}
\usepackage{hyperref}
\usepackage{fancyhdr}
\usepackage{graphicx}
\usepackage[export]{adjustbox}
\usepackage{t1enc}
\usepackage[english]{babel}
\usepackage{bm}
\usepackage{multirow}

\usepackage{booktabs}

\usepackage{pgfplots}
\pgfplotsset{height = 10cm, width=15cm,compat=1.9}

% \usepackage[usenames,dvipsnames]{xcolor}
\usepackage[left=2cm,right=2cm,top=2cm,bottom=2cm]{geometry}

\usepackage{listings} % For inline code listings
\usepackage{xcolor}   % For custom colors in listings

% Define Python code style for listings
\lstdefinestyle{python}{
    language=Python,
    basicstyle=\ttfamily\small,
    keywordstyle=\color{blue}\bfseries,
    stringstyle=\color{red},
    commentstyle=\color{gray},
    showstringspaces=false,
    frame=single,
    numbers=left,
    numberstyle=\tiny\color{gray},
    breaklines=true,
    breakatwhitespace=true,
    tabsize=4
}

\setlength{\parindent}{0pt} % bekezdés behúzása
\setlength{\parskip}{0em}   % bekezdések közti távolság
\pagestyle{fancy}
\fancyhf{}


% --> disable section num
% \setcounter{secnumdepth}{0}



\title{Finite elastic-plastic deformations\\(BMEGEMMDKPL)\\I. Homework}
\author{Szász Zsolt\\KRCH5Q}
\date{April 4, 2025}

\lhead{Szász Zsolt\\KRCH5Q}
\chead{}
\rhead{Finite elastic-plastic deformations\\I. Homework}
\cfoot{\thepage. page}

% ITT KEZDŐDIK A DOKUMENTUM



\begin{document}


\maketitle{}
\includepdf[pages=-]{HF.pdf}
\newpage

\section{Determination of the Prony parameters}

The Prony parameters are used to describe the viscous behavior of materials. We can determine them by fitting the viscoelastic model's solution to the DMA experiment data. The measured quantities are the storage modulus $E'$ and the loss modulus $E''$, as shown in the Figure \ref{fig:experiment}.


\begin{figure}[h]
    \centering
    \includegraphics[scale=0.875]{figures/data.png}
    \caption{Virtual DMA experiment data}
    \label{fig:experiment}
\end{figure}

We know that the storage modulus approaches the instantaneous modulus as the frequency increases, therefore the instantaneous modulus can be read from the figure as

\begin{equation}
    \lim_{\omega\rightarrow\infty}E'(\omega) = E_0 \approx 45\;\text{MPa}.
\end{equation}

If relaxation characteristic is described with an $N$th order Prony-series, the relative storage modulus $E'/E_0$ and the relative loss modulus $E''/E_0$ can be expressed as a function of the frequency $\omega$ as

\begin{equation}
\frac{E'(\omega)}{E_0} = 
1 - \sum_{i=1}^N \frac{1}{1+\tau_i^2 \omega^2}e_i,
\end{equation}

\begin{equation}
\frac{E''(\omega)}{E_0} = 
\sum_{i=1}^N \frac{\tau_i \omega}{1+\tau_i^2 \omega^2}e_i,
\end{equation}

where the $\{e_i\}_{i=1}^N$ and $\{\tau_i\}_{i=1}^N$ are the unknown Prony parameters. We will fit these functions on the measurement data, by optimizing the following objective function:

\begin{equation}
    Q\left(\{e_i\}_{i=1}^N,\{\tau_i\}_{i=1}^N\right) = 
\frac{1}{n}\sum_{j=1}^n \left[\left(1-\frac{E'(\omega_j)}{E'_j}\right)^2 + \left(1-\frac{E''(\omega_j)}{E''_j}\right)^2\right].
\end{equation}


I used the \texttt{scipy.optimize.minimize} method for this purpose:  

\lstset{style=python}
\begin{lstlisting}
def fit_prony(N):
    # Initial values
    e = [E_storage_norm_true[0]/2] * N
    tau = [100] * N
    initial_guess = e + tau
    # Conditions
    bounds = [(0, None)] * len(initial_guess)
    # Optimization
    result = minimize(Q, initial_guess, bounds=bounds)
    return result.x[0:N], result.x[N:], result.fun
\end{lstlisting}

\newpage

We can repeat the optimization for different orders of the Prony series. On Figure \ref{fig:prony_order}. we can see how the final value of the objective function depends on the order of the choosen order of the Prony series.

\begin{figure}[H]
    \centering
    \includegraphics[scale=0.9]{figures/best_prony.png}
    \caption{The value of the objective function as a function of the order of the Prony series}
    \label{fig:prony_order}
\end{figure}

We can observe that the 4th and the 8th order Prony series gives the best fit. The 9th order is a bit better, but it is not worth to use such a high order, so we will use the 4th order one. The fitted 4th order Prony parameters are summarized in  Table \ref{tab:prony_parameters}.

\begin{table}[H]
    \centering
    \caption{Fitted Prony Parameters}
    \label{tab:prony_parameters}
    \begin{tabular}{|c|c|c|}
        \hline
        \textbf{i} & \textbf{Value of $e_i$} & \textbf{Value of $\tau_i$} \\ \hline
        $1$              & 0.1000             & 50.0052                    \\ \hline
        $2$              & 0.2000              &  1.0001                   \\ \hline
        $3$              & 0.1000              & 10.0020                    \\ \hline
        $4$              & 0.4000              & 0.1000                    \\ \hline
    \end{tabular}
\end{table}

We can also plot the fitted storage and loss moduli against the measured data. The fitted curve is shown in Figure \ref{fig:fittedmodel}.
\begin{figure}[H]
    \centering
    \includegraphics[scale=0.875]{figures/fitted.png}
    \caption{Fitted storage and loss moduli}
    \label{fig:fittedmodel}
\end{figure}

The value of the objectife function for this fit is $Q = 2.8\times 10^{-9}$, we accept this as a good fit.

\newpage

We have an incpomressible material, which means the deviatoric part of the stress determines the volumetric part, therefore we will assume that the relaxation characteristic defined by the previously calculated Prony parameters, belongs to the deviatoric behaviour, which results

\begin{equation}
    \{g_i\}_{i=1}^N = \{e_i\}_{i=1}^N.
\end{equation}

\section{The instantaneous stress response}

The main behaviour of the material is described using a first-order Ogden-type hyperelastic model, which can be expressed as 

\begin{equation}
    W(\lambda_1, \lambda_2, \lambda_3) = \frac{2\mu_1}{\alpha_1^2} \left( \lambda_1^{\alpha_1} + \lambda_2^{\alpha_1} + \lambda_3^{\alpha_1} - 3 \right), 
\end{equation}

We know that uniaxial loading is prescirbed. The solution of the uncompressible Ogden model for the uniaxial loading was shown at continuum mechanics lectures. The Cauchy stress solution can be expressed as

\begin{equation}
\sigma = \frac{2\mu_1}{\alpha_1}(\lambda^{\alpha_1} - \lambda^{-\frac{\alpha_1}{2}}),
\end{equation}

where $\mu_1 = 3$ [Mpa] and $\alpha_1 = 0.5$ are given for the infinitely slow loading. However, for our purposes (specifying the viscoelastic solution) we need the parameters corresponding to infinitely fast loading. We can linearize the expression around $\lambda = 1$ like

\begin{equation}
    \sigma = \frac{2\mu_1}{\alpha_1}(\lambda^{\alpha_1} - \lambda^{-\frac{\alpha_1}{2}}) \approx \frac{2\mu_1}{\alpha_1} \left(\alpha_1 + \frac{\alpha_1}{2}\right) (\lambda - 1) = 3\mu_1 \varepsilon \Rightarrow E = 3 \mu_1.
    \end{equation}

Now we can calculate the investigated $\mu_1$ for the infinitely fast loading case too

\begin{equation}
    \mu_1 = \frac{E_0}{3} = 15 \;[\text{MPa}]
\end{equation}

From the specified loading, the uniaxial stretch can be expressed as

\begin{equation}
\lambda(t) = \varepsilon(t) + 1 = 15t\cdot e^{-2t}-0.09t + 1
\end{equation}

Substituting into the Cauchy stress equation, we can obtain the time evolution of the Cauchy stress. The result is shown in Figure \ref{fig:instantaneous_stress}.
\begin{figure}[H]
    \centering
    \includegraphics[scale=0.8]{figures/elastic_solution.png}
    \caption{Instantaneous stress response}
 \label{fig:instantaneous_stress}
\end{figure}

\newpage

\section{The viscoelastic stress response}

For uniaxial loading, the deviatoric part of the instantaneous Cauchy stress tensor can be expressed as

\begin{equation}
    \boldsymbol{s}_0(t) = 
    \sigma_0(t)
    \begin{bmatrix}
        2/3 & 0 & \\
        0 & -1/3 & 0\\
        0 & 0 & -1/3\\
    \end{bmatrix}
\end{equation}


We know from the lecture that, the deviatoric part of the viscoelastic stress can be expressed as

\begin{equation}
\boldsymbol{s}(t) = \boldsymbol{s}_0(t) - 
\text{dev}\left[
    \boldsymbol{F}(t)
        \left(
            \sum_{i=1}^N \frac{e_i}{\tau_i} \int_0^t \boldsymbol{F}^{-1}(t-s) \boldsymbol{s}_0(t-s) \boldsymbol{F}^{-T}(t-s)e^{-\frac{s}{\tau_i}}\text{d}s
        \right)
    \boldsymbol{F}^T(t)
\right].
\end{equation}

We know that the deformation gradient for uniaxial extension looks like

\begin{equation}
\boldsymbol{F}(t) = \boldsymbol{F}^T(t) = 
\begin{bmatrix}
    \lambda(t) & 0 & 0\\
    0&\lambda^{-1/2}(t)&0\\
    0&0&\lambda^{-1/2}(t)\\
\end{bmatrix},
\end{equation}

and it's inverse looks like

\begin{equation}
\boldsymbol{F}^{-1}(t) =  \boldsymbol{F}^{-T}(t)=
\begin{bmatrix}
    \lambda^{-1}(t) & 0 & 0\\
    0&\lambda^{1/2}(t)&0\\
    0&0&\lambda^{1/2}(t)\\
\end{bmatrix}.
\end{equation}

Everything is given to determine the deviatoric part of the stress. The only thing we need to do is to choose a numerical method to solve the convolutional integral. I used simple discrete convolution:

\lstset{style=python}
\begin{lstlisting}
# self defined convolution function
def conv(f,g,t):
    dt = t[1] - t[0]
    ret = [
        np.sum(f(t[k] - t[:k]) * g(t[:k]) * dt,axis=-1) 
        for k in range(len(t))
    ]
    return np.array(ret).T

# placeholder for the convolution result
SUM = np.zeros((3,len(t)))

# calculate the convolution terms
for i in range(len(tau)):
    f = lambda t: F_inv(t)*s0(t)*F_inv(t)
    g = lambda t: np.e**(-t/tau[i])
    SUM += e[i]/tau[i] * conv(f,g,t)

# deviatoric stress
s = s0(t) - dev(F(t)*SUM*F(t))
\end{lstlisting}


\newpage

Unfortunately, if we check the final values of the stress, we can observe that, with this method we need a very high temporal resolution to achieve the accuracy required by the task, since the $dt=10^{-5}$ [s] is not enough. 

At the same time, the general convolution with  $dt=10^{-6}$ [s] has too high computational cost. Therefore, alternative numerical methods or optimized algorithms are required to balance accuracy and efficiency. I will just replace the the convolution function, with an FFT accelerated specral convoltion method, known as \texttt{scipy.signal.fftconvolve}. The modified function is:

\lstset{style=python}
\begin{lstlisting}
from scipy.signal import fftconvolve

def conv(f, g, t):
    dt = t[1] - t[0]
    T = len(t)

    f_vals = f(t)
    g_vals = g(t)

    res = np.array([
        fftconvolve(f_vals[j, :], g_vals, mode='full')[:T] * dt
        for j in range(3)
    ])
    return res
\end{lstlisting}


We can plot the deviatoric stress response as a function of time. The result is shown in Figure \ref{fig:viscoelastic_stress}.
\begin{figure}[H]
    \centering
    \includegraphics[scale=1]{figures/deviatoric_response.png}
    \caption{Deviatoric part of the stress response}
    \label{fig:viscoelastic_stress}
\end{figure}

The volumetric part of the stress can be calculated from the boundary conditions in the following way:

\begin{equation}
    \begin{matrix}
        \sigma_{22}(t) = s_{22}(t) + p(t) = 0, \\
        \sigma_{33}(t) = s_{33}(t) + p(t) = 0.
    \end{matrix}
\end{equation}

Therefore, we can express the volumetric part of the stress as
\begin{equation}
    \boldsymbol{p}(t) = - s_{22}(t)\boldsymbol{I}.
\end{equation}

\newpage

The viscoelastic stress response can be expressed as

\begin{equation}
    \boldsymbol{\sigma}(t) = \boldsymbol{s}(t) + \boldsymbol{p}(t).
\end{equation}

We can be plot the result as the function of the time and the function of the strech. The result is shown in Figure \ref{fig:stress}.

\begin{figure}[H]
    \centering
    \includegraphics[scale=0.8]{figures/stress.png}
    \caption{Viscoelastic stress response}
    \label{fig:stress}
\end{figure}

Furthermore we can print the final value of the stress response: $\boldsymbol{\sigma}(t=10) = -46.10578$ [Mpa].


\section*{References}

You can find the detailed code on \href{https://github.com/zsoca000/Finite-elastic-deformations-HW1/blob/main/solution2.ipynb}{\textcolor{blue}{\underline{GitHub}}}.

\end{document}